% !TEX root = ./main.tex
\documentclass[UTF8]{ctexart}                       % 中文文档
\usepackage{fancyhdr}                               % 页眉页脚
\fancyhead{}                                        % 清空页眉
\fancyhead[R]{\thepage}                             % 在页眉右侧显示当前页数
\usepackage{amsmath, amssymb}                       % 必要的数学环境、字体与符号
\usepackage{fontspec}                               % 字体处理
\usepackage{listings}                               % 代码片段插入
\usepackage{graphicx}                               % 图片处理
% \usepackage{float}                                  % 允许将浮动体放置在指定的位置([H]),而不受LaTeX的默认浮动机制控制
\usepackage{caption}                                % 标题处理,可以用这个宏包的caption*命令实现图片无编号插入
\usepackage{xcolor}                                 % 颜色处理
\usepackage[colorlinks,linkcolor=violet]{hyperref}  % 超链接插入
\usepackage{array}                                  % 提供额外的列类型选项
% \definecolor{mGreen}{rgb}{0,0.6,0}
% \definecolor{mGray}{rgb}{0.5,0.5,0.5}
% \definecolor{mPurple}{rgb}{0.58,0,0.82}
% \definecolor{backgroundColour}{rgb}{0.95,0.95,0.92}
% \newfontfamily\cc{CASCADIACODE.TTF}                 % 选择CascadiaCode作为代码片段的字体
% \lstdefinestyle{CppStyle}{
%     backgroundcolor=\color{backgroundColour}, 
%     commentstyle=\color{mGreen},
%     keywordstyle=\color{magenta},
%     numberstyle=\tiny\color{mGray},
%     stringstyle=\color{mPurple},
%     basicstyle=\small\cc,
%     escapeinside={\%*}{*)},                         % 用%*和*)在代码片段中插入LaTeX语法
%     breakatwhitespace=false,         
%     breaklines=true,                 
%     captionpos=b,                    
%     keepspaces=true,                 
%     numbers=left,                                   % 在代码片段左侧显示行数                 
%     numbersep=5pt,                  
%     showspaces=false,                
%     showstringspaces=false,
%     showtabs=false,                  
%     tabsize=2,
%     language=C++
% }

\newcommand{\courseName}{计算机网络基础}
\newcommand{\courseTeacher}{张三浦}
\newcommand{\myName}{李田所}
\newcommand{\myFacultad}{计算机科学与技术学院}
\newcommand{\myMajor}{信息安全}
\newcommand{\myStudentID}{3210114514}
\newcommand{\coworkerName}{}
\newcommand{\labType}{操作实验}
\newcommand{\labLocation}{计算机网络实验室}
\newcommand{\labName}{WireShark软件初探和常见网络命令的使用}

\ctexset{
    % 设置章节标题左对齐
    section = {
        format = \Large\bfseries
    }
}

\AtBeginDocument{
    % 设置equation与align环境与上下文之间的空隙,使之更加紧凑
    \setlength{\abovedisplayskip}{-10pt}
    \setlength{\belowdisplayskip}{5pt}
    % 设置equation*与align*环境与上下文之间的空隙,使之更加紧凑
    \setlength{\abovedisplayshortskip}{-15pt}
    \setlength{\belowdisplayshortskip}{0pt}
}
\begin{document}

\begin{titlepage}

    \begin{figure}[h]
        \centering
        \includegraphics[width=0.66\textwidth]{../image/logo.png}
    \end{figure}

    \begin{center}
        \huge{\textbf{本科实验报告\\}}
    \end{center}

    \vfill
    \begin{center}
        \Large  % \Large 每个汉字约等于0.5cm
        \renewcommand{\arraystretch}{1.25}
        \begin{tabular}{>{\raggedright}p{3cm}p{8cm}}
            \multicolumn{1}{c}{课程名称:}          & \multicolumn{1}{c}{\underline{\makebox[8cm][l]{\courseName}}}     \\
            \multicolumn{1}{c}{姓\hspace{1cm}名:} & \multicolumn{1}{c}{\underline{\makebox[8cm][l]{\myName}}}          \\
            \multicolumn{1}{c}{学\hspace{1cm}院:} & \multicolumn{1}{c}{\underline{\makebox[8cm][l]{\myFacultad}}}      \\
            \multicolumn{1}{c}{系\hspace{1cm}别:} & \multicolumn{1}{c}{\underline{\makebox[8cm][l]{}}}                 \\
            \multicolumn{1}{c}{专\hspace{1cm}业:} & \multicolumn{1}{c}{\underline{\makebox[8cm][l]{\myMajor}}}         \\
            \multicolumn{1}{c}{学\hspace{1cm}号:} & \multicolumn{1}{c}{\underline{\makebox[8cm][l]{\myStudentID}}}     \\
            \multicolumn{1}{c}{指导教师:}          & \multicolumn{1}{c}{\underline{\makebox[8cm][l]{\courseTeacher}}}  \\
        \end{tabular}
    \end{center}

    \vfill
    \centerline{\Large{\today}}
    \vfill

\end{titlepage}

\newpage

\begin{center}
    \Large{\textbf{浙江大学实验报告\\}}
\end{center}

\begin{center}
    % 没有任何修正的情况下每个汉字约等于0.4cm,每行能够容纳32个汉字
    \begin{tabular}{>{\raggedright}p{2cm}p{3.66cm}p{2cm}p{3.66cm}}
        \multicolumn{1}{c}{课程名称:} & \multicolumn{1}{c}{\underline{\makebox[3.66cm][c]{\courseName}}}   &
        \multicolumn{1}{c}{实验类型:}  & \multicolumn{1}{c}{\underline{\makebox[3.66cm][c]{\labType}}}     \\
    \end{tabular}
    \begin{tabular}{>{\raggedright}p{3cm}p{9cm}}
        \multicolumn{1}{c}{实验项目名称:} & \multicolumn{1}{c}{\underline{\makebox[9cm][c]{\labName}}}     \\
    \end{tabular}
    \begin{tabular}{>{\raggedright}p{2cm}p{2cm}p{1.2cm}p{2.4cm}p{1.2cm}p{2cm}}
        \multicolumn{1}{c}{学生姓名:}  & \multicolumn{1}{c}{\underline{\makebox[2cm][c]{\myName}}}             &
        \multicolumn{1}{c}{专业:}     & \multicolumn{1}{c}{\underline{\makebox[2.4cm][c]{\small{\myMajor}}}}   &
        \multicolumn{1}{c}{学号:}     & \multicolumn{1}{c}{\underline{\makebox[2cm][c]{\myStudentID}}}         \\
    \end{tabular}
    \begin{tabular}{>{\raggedright}p{3cm}p{4.6cm}p{2cm}p{2cm}}
        \multicolumn{1}{c}{同组学生姓名:} & \multicolumn{1}{c}{\underline{\makebox[4.6cm][c]{\coworkerName}}}  &
        \multicolumn{1}{c}{指导老师:}  & \multicolumn{1}{c}{\underline{\makebox[2cm][c]{\courseTeacher}}}      \\
    \end{tabular}
    \begin{tabular}{>{\raggedright}p{2cm}p{3.66cm}p{2cm}p{3.66cm}}
        \multicolumn{1}{c}{实验地点:} & \multicolumn{1}{c}{\underline{\makebox[3.65cm][c]{\labLocation}}}      &
        \multicolumn{1}{c}{实验日期:}  & \multicolumn{1}{c}{\underline{\makebox[3.65cm][c]{\today}}}           \\
    \end{tabular}
\end{center}

\vspace{0.5cm}

\sloppy % 在行尾自动断字,而不考虑语义、单词形状或结构
\raggedbottom % 允许页面底部的空白区域长度不同,从而使页面之间的垂直间距保持一致

% !TEX root = ./main.tex

% 如果有每一章节分页的需求,可以使用这个命令强制将所有浮动体排版出来,并开始一个新的页面。
% \clearpage

\section{实验目的和要求}

\begin{itemize}
    \item 初步了解WireShark软件的界面和功能
    \item 熟悉各类常用网络命令的使用
\end{itemize}


% !TEX root = ./main.tex

\section{实验内容和原理}

\begin{itemize}
    \item Wireshark是PC上使用最广泛的免费抓包工具,可以分析大多数常见的协议数据包。有Windows版本、Linux版本和Mac版本,可以免费从网上下载 
    \item 初步掌握网络协议分析软件Wireshark的使用,学会配置过滤器
    \item 根据要求配置Wireshark,捕获某一类协议的数据包
    \item 在PC机上熟悉常用网络命令的功能和用法: Ping.exe,Netstat.exe, Telnet.exe, Tracert.exe, Arp.exe, Ipconfig.exe, Net.exe, Route.exe, Nslookup.exe
    \item 利用WireShark软件捕捉上述部分命令产生的数据包
\end{itemize}


% !TEX root = ./main.tex

\section{主要仪器设备}

\begin{itemize}
    \item 联网的PC机
    \item WireShark协议分析软件
\end{itemize}


% !TEX root = ./main.tex

% 如果有每一章节分页的需求,可以使用这个命令强制将所有浮动体排版出来,并开始一个新的页面。
% \clearpage

\section{操作方法与实验步骤}

\begin{enumerate}
    \item 项目1
    \begin{enumerate}
        \item 子项目1
        \item 子项目2
    \end{enumerate}
    \item 项目2
    \item 项目3
\end{enumerate}

% !TEX root = ./main.tex

% 如果有每一章节分页的需求,可以使用这个命令强制将所有浮动体排版出来,并开始一个新的页面。
% \clearpage

\section{实验数据记录和处理}

\subsection{问题1}

\par 回答1


% !TEX root = ./main.tex

\section{实验结果与分析}

\subsection{问题1}

\par 回答1

% !TEX root = ./main.tex

% 如果有每一章节分页的需求,可以使用这个命令强制将所有浮动体排版出来,并开始一个新的页面。
% \clearpage

\section{讨论、心得}

\par \textcolor{red}{实验过程中遇到的困难,得到的经验教训,对本实验安排的更好建议(看完请删除本句)}

% 代码示例,需要取消注释main.tex关于lstdefinestyle的定义
% 你可以参考lstlisting的手册查看受支持的编程语言,然后添加你需要的语言对应的style
%
% \begin{lstlisting}[style=CppStyle]
% int WPL = 0;
% priority_queue<int, vector<int>, greater<int>> qord;
% while (qord.size() != 1)
% {
%     int a, b;
%     a = qord.top();
%     qord.pop();
%     WPL += a;
%     b = qord.top();
%     qord.pop();
%     WPL += b; 
%     qord.push(a + b);
% }
% \end{lstlisting}

\end{document}

